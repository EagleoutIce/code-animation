\documentclass[handout]{beamer}
\usepackage{code-animation}

\errorcontextlines9999
\usepackage[encoding, defaultfont, fakeminted]{sopra-listings}
\solsetmintedstyle{plain}

% TODO: handout mode
\begin{document}
\begin{frame}[fragile]
\AnimateCode{onslide={4, % just a highlight
   :1:{6}, % : allows to specify others to highlight
   /2: Nun ist \T{i = 9}, % highlight with comment
   /3: Hat jemand einen Pinguin gesehen?,
   /4: $3x^2$, .5, % basic highlight which rests comments
   /2: Nun ist wieder \T{i = 9}, /3: Pinguuuu, 4,
   |5: Ein neues leben!,% comments which resets other comments
   3,3, 0, % Dopplungen sind kein Problem, 0 shows full code
   *\Line{4}\Comment{Walter}\Location{end},% komplett frei :D
   3, .4
},first slide=2}
\begin{minted}{java}
public static void main(String[] args) {
   int i = 9;
   char x = 'c';
   String k = "k";
   System.out.println(x + ": " + i);
}
\end{minted}
\endAnimateCode
\end{frame}

\begin{frame}
   buffer
\end{frame}

\begin{frame}[fragile]
\AnimateCode{onslide={1, % just a highlight
   /2: Nun ist \T{i = 12}, % highlight with comment
   /3: Hat jemand einen Pinguin gesehen?,
   /4: $3x^2$,
   .5, % basic highlight which rests comments
   /2: Nun ist wieder \T{i = 12},
   /3: Pinguuuu,
   4,
   |5: Ein neues leben!,% comments which resets other comments
   3,3, % auch kein Problem
   0, % show full code
   *\Line{4}\Comment{Walter}\Location{end},% komplett frei :D
   3, .4
},first slide=2}
\begin{minted}{java}
public static void main(String[] args) {
   int i = 12;
   char x = 'c';
   String k = "k";
   System.out.println(x + ": " + i);
}
\end{minted}
\endAnimateCode
\end{frame}

\begin{frame}[fragile]
\begin{minted}{java}
public static void main(String[] args) {
   int i = 9;
   char x = 'c';
   System.out.println(x + ": " + i);
}
\end{minted}
\end{frame}
\end{document}