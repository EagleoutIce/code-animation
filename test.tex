\documentclass{beamer}
\usepackage{code-animation}

% TODO: keep ghost markers for recursion (similar to comments)
% todo: clear single comments
% todo: elements without anim?
\errorcontextlines9999
\usepackage[encoding, defaultfont, fakeminted]{sopra-listings}
\solsetmintedstyle{plain}
\begin{document}
% \CodeAnimationsNoFadeOut
\begin{frame}[fragile]
\AnimateCode{onslide={%
   % o would allow a negative number which would be the same as no highlight
   o3:{4,5},-,-, % only highlight no line marker
   2,+,+,+, % just a highlight
   :1:{6}, % : allows to specify others to highlight
   /2: Nun ist \T{i = 9}, % highlight with comment
   /3: Hat jemand einen Pinguin gesehen?,
   /4: $3x^2$, .5, % basic highlight which rests comments
   /2: Nun ist wieder \T{i = 9}, /3: Pinguuuu, 4,
   |5: Ein neues leben!,% comments which resets other comments
   3,3, 0, % Dopplungen sind kein Problem, 0 shows full code
   *\Line{4}\Comment{Walter}\Location{end}\StoreAnimationTo\WalterAnim\StoreHandoutTo\WalterHandout\StoreTo\Walter,
   *\Line{4}\Comment{Custom!}\Location{k}\Reset,% komplett frei :D
   3, .4
},handout={8/2,\CodeAnimGet{WalterAnim}/3},first slide=5}% CodeAnimGet can be used pre definition
\begin{minted}[escapeinside={/*}{*/}]{java}
/*\onslide<2->*/public static void main(String[] args) {
/*\onslide<3->*/   int i = 9;
/*\onslide<3->*/   char x = 'c';
/*\onslide<3->*/   String /*\AnimLoc[2pt]{k}*/k = "k";
/*\onslide<4->*/   System.out.println(x + ": " + i);
/*\onslide<2->*/}
\end{minted}
\endAnimateCode
Walter kommt (nur bei Animationen natürlich) auf Slide \WalterAnim~(\CodeAnimGet{WalterAnim}). Sein Handout-Mapping ist Folie \WalterHandout~(\CodeAnimGet{WalterHandout}). Das Auto-Mapping ist \Walter~(\CodeAnimGet{Walter}). Markername: \GetAnimLoc{k}. % TODO: improve handling of symbolic stuff
\end{frame}

\begin{frame}
   buffer
\end{frame}

\begin{frame}[fragile]
\AnimateCode{onslide={1, % just a highlight
   /2: Nun ist \T{i = 12}, % highlight with comment
   /3: Hat jemand einen Pinguin gesehen?,
   /4: $3x^2$,
   .5, % basic highlight which rests comments
   /2: Nun ist wieder \T{i = 12},
   /3: Pinguuuu,
   4,
   |5: Ein neues leben!,% comments which resets other comments
   3,3, % auch kein Problem
   0, % show full code
   *\Line{4}\Comment{Walter}\Location{end},% komplett frei :D
   3, .4
},first slide=2}
\begin{minted}{java}
public static void main(String[] args) {
   int i = 12;
   char x = 'c';
   String k = "k";
   System.out.println(x + ": " + i);
}
\end{minted}
\endAnimateCode
\end{frame}

\begin{frame}[fragile]
\begin{minted}{java}
public static void main(String[] args) {
   int i = 9;
   char x = 'c';
   System.out.println(x + ": " + i);
}
\end{minted}
\end{frame}
\end{document}